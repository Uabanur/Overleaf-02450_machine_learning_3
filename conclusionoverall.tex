\subsection{Conclusion on the project}
The tasks outlined for the 3 reports on the project successfully encompassed all of the learning goals of course \textcolor{blue}{02450 Introduction to Machine Learning and Data Mining}\footnote{\url{http://kurser.dtu.dk/course/02450}} through the use of fundamental and widely applied methods for machine learning and data modelling.

\textbf{In report 1}, feature extraction and visualization were performed to give a basic understanding of the data set. The data was standardized and visualized through Principal Component Analysis. A \textcolor{Bittersweet}{\textit{regression problem}} and a \textcolor{Bittersweet}{\textit{classification problem}} were presented, to be solved in report 2: predicting refraction index from chemical weight attributes (\textcolor{Bittersweet}{\textit{regression}}) and predicting the glass type (\textcolor{Bittersweet}{\textit{classification}}).

\textbf{In report 2}, key machine learning concepts such as cross-validation, generalization and over-fitting and prediction were addressed. The supervised learning problems presented report 1 were solved through various standard machine learning methods like linear regression, $K$-nearest neighbors, decision trees and artificial neural networks.

\textbf{In report 3}, the focus was on unsupervised learning methods: density estimation, clustering and association mining. Patterns were discovered and compared to the labelled data from report 2. Outliers/anomalies were discovered and addressed.

The experience from working on the Glass Identification data set, and the knowledge gained on the data modeling framework, generalizes to a broad range of application domains like bio-informatics, electrical engineering and computer science, so the 3 reports and course 02450 has enabled us to apply machine learning for modeling of real world data and prepared us for future courses like \textcolor{blue}{02457 Non-Linear Signal Processing} and \textcolor{blue}{02460 Advanced Machine Learning}.


