Detecting anomalies is a way to gain information about the respective data set. An anomaly or outlier is often defined as an observation which deviates from the other observations. This is a very intuitive interpretation of an anomaly, but when is the observation deviating enough to go in this category? Three scoring methods are represented in the following subsection in order to address this. All three methods make use of the density of the observations in the data set, where high density corresponds to observations close together in the $M$-dimensional space defined by the attributes of the observations. The used attributes are the glass chemical composition percentages and the refractive index. To visualize the potential outliers, found from each method, the data set is projected onto the two most principal components in the final discussion of the section. Formulas are based on section 18 from the course notes \cite{coursenotes}. 